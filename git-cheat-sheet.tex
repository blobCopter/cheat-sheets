\documentclass{article}

% Shell lines
\usepackage{xcolor}
\usepackage{listings}
\lstdefinestyle{BashInputStyle}{
  language=bash,
  basicstyle=\small\sffamily,
  numbers=left,
  numberstyle=\tiny,
  numbersep=3pt,
  frame=tb,
  columns=fullflexible,
  backgroundcolor=\color{yellow!20},
  linewidth=0.9\linewidth,
  xleftmargin=0.1\linewidth
}

\title{Git Cheat Sheet}
\author{Patrick Rabier}

\begin{document}
\maketitle

%%%%%%%%%%%%%%%%%%%%%%%%%%%%%%%%%%%%%%%%%%%%

\section{Git Basics}

\subsection{Adding Changes to the repository}
After creating or modifying files, add them to the index with the following command :
\begin{lstlisting}[style=BashInputStyle]
    #  git add [files]
\end{lstlisting}

The index contains all the files that will be commited

\subsection{Commiting files}

Commit the files which have been added to the index with the following :

\begin{lstlisting}[style=BashInputStyle]
    #  git commit -m [Commit message]
\end{lstlisting}

The '-a' flag can be used to automatically add all the versioned files which have been modified.

\subsection{Pushing the changes}

To send the local changes to the remote server, use the push method as follows: 

\begin{lstlisting}[style=BashInputStyle]
    #  git push
\end{lstlisting}

%%%%%%%%%%%%%%%%%%%%%%%%%%%%%%%%%%%%%%%%%%%%

\section{Git Branching}

How to manipulate branches

\subsection{Displaying existing branches}
\begin{lstlisting}[style=BashInputStyle]
    #  git branch
\end{lstlisting}

\subsection{Creating a branch}
Use the following command to create a branch :
\begin{lstlisting}[style=BashInputStyle]
    # git branch [branch name]
\end{lstlisting}
This only creates the branch. It does not switch to it

\subsection{Switching to a branch}
\begin{lstlisting}[style=BashInputStyle]
    #  git checkout [branch name]
\end{lstlisting}
Modifications you make on a branch won't affect other branches

\subsection{Pushing a branch to a remote repository}
\begin{lstlisting}[style=BashInputStyle]
    # git push [repository] [branch name]
\end{lstlisting}
If the branch doesn't exist on the remote repository, it will be created.
If the branch already exists, it will be updated with the latest commits
Example :
\begin{lstlisting}[style=BashInputStyle]
    # git push origin mybranch
\end{lstlisting}
'origin' is typically the main repository. However, others may be added using the following command :
\begin{lstlisting}[style=BashInputStyle]
    # git remote add [repository name] [repository url]
\end{lstlisting}

\subsection{Deleting a local branch}
\begin{lstlisting}[style=BashInputStyle]
    #  git branch -d [branch name]
\end{lstlisting}
This will only delete the branch locally

\subsection{Deleting a remote branch}
\begin{lstlisting}[style=BashInputStyle]
    #  git push [remote repository name] :[branch name]
\end{lstlisting}
This will only delete the branch on the remote repository
Example :
\begin{lstlisting}[style=BashInputStyle]
    #  git push origin :mybranch
\end{lstlisting}

%%%%%%%%%%%%%%%%%%%%%%%%%%%%%%%%%%%%%%%%%%%%

\section{Fixing Sh** up}

\subsection{Reverting a commit after a mistake was made}

\begin{lstlisting}[style=BashInputStyle]
    #  git revert HEAD~1
\end{lstlisting}

if the bad commit has already been pushed, it can be overriden by forcing the push with the '-f' flag, carefully of course.

\subsection{Conflicts in a pull request ?}

Rebase your branch on the one you're trying to merge in

\begin{lstlisting}[style=BashInputStyle]
    #  git rebase [branch to merge into]
\end{lstlisting}

This will apply every commit you have made on the branch. When conflicts occur while rebasing, simply fix the files and add them

\begin{lstlisting}[style=BashInputStyle]
    #  git add [files]
\end{lstlisting}

After adding the files, the rebase can continue

\begin{lstlisting}[style=BashInputStyle]
    #  git rebase --continue
\end{lstlisting}

%%%%%%%%%%%%%%%%%%%%%%%%%%%%%%%%%%%%%%%%%%%%


\section{Simply useful}

\subsection{Git Stash}

\subsubsection{Stashing your changes}

If you have made changes to a file that you do not with to have any more, but still want to save for later, you can stash them with the following command

\begin{lstlisting}[style=BashInputStyle]
    #  git stash
\end{lstlisting}

This will add them to a stack of changes.

\subsubsection{Re-using your stashed changes}

You can pop the stack of changes, and have the latest changes be applied using the following command :

\begin{lstlisting}[style=BashInputStyle]
    #  git stash pop
\end{lstlisting}

If you wish to apply them, but without popping the stack, just use the 'apply' command as follows :

\begin{lstlisting}[style=BashInputStyle]
    #  git stash apply
\end{lstlisting}

\subsubsection{Stashing only a portion of your changes}

\begin{lstlisting}[style=BashInputStyle]
    #  git stash -p
\end{lstlisting}


%%%%%%%%%%%%%%%%%%%%%%%%%%%%%%%%%%%%%%%%%%%%

\end{document}
